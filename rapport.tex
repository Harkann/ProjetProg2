\documentclass{article}
\usepackage[utf8]{inputenc}
\usepackage[T1]{fontenc}
\usepackage{amsmath}
\usepackage{amssymb}
\title{Projet Programmation 2}
\date{}
\author{Garance GOURDEL, Michaël PAULON}
\begin{document}
\maketitle
\section{Introduction}
L'objectif du projet était de réaliser un jeu d'échec en programmation orientée objet dans le langage Scala. \\
Notre projet est divisé comme suit : \\
 - partie.scala \\
 - interface.scala \\
 - pieces.scala \\

\section{partie.scala}
Gère le fonctionnement de la partie et l'ia\\
Le plateau a été représenté par une matrice contenant l'identité des pièces. \\
Le tour de l'IA se déroule dans un nouveau thread pour permettre de rajouter du délai entre ses actions sans impacter le reste du jeu.



\section{interface.scala}
Gère l'interface \\
Le joueur n'est pas un objet à part entière, mais ses actions sont entièrement codées par les boutons.\\

\section{pieces.scala}
\subsection{Déplacements des pièces}
Pour factoriser le code, les déplacement sont codés sous forme de traits : \\
 - horizontaux et verticaux pour la tour et la reine \\
 - diagonaux pour le fou et la reine \\
 - jump pour les cavaliers \\
 - move\_peon pour les pions \\
 - move\_king pour le roi \\

\subsection{Identification des pièces}
L'identité d'une pièce est crée ainsi : \\ 
 - un caractère pour la couleur de la pièce, \\
 - deux caractères pour le nom de la pièce \\
 - un chiffre pour le numéro de la pièce \\
Ainsi le quatrième pion noir a pour id : "BPe3" \\

\section{Répartition du travail}
interface.scala : Michaël \\
pieces.scala : Garance \\
partie.scala : \\
 - vérification de la mise en échec et du mat : Garance \\
 - IA et mouvements autorisés : Michaël \\
 \end{document}